\section{Introduction}

Numbat is a simulation model for IEEE 802.16 \cite{802.16e}, better
known under its commercial name Mobile WiMax. As its original purpose
was to simulate IPv6 mobile node, with focus on the (re)configuration
processes, stateless (RA, DAD) and stateful (DHCPv6) configuration
methods are also part of this model. 

Following features are supported in the WiMAX layer:
\begin{itemize}
\item based on IEEE 802.16-2005 (aka 802.16e)
\item Simple radio/PHY layer - unicast (SS to BS) and multicast (BS to SS) transmissions
\item OFDMA transmission (including CDMA codes, radio frame slots/symbols)
\item Bandwidth management: BWR, CDMA codes for ranging and BWR transmission)
\item Supported traffic classes: Best Effort (BE) and Unsolicited Grant Service (UGS)
\item Control plane: Network entry(RNG-REQ, RNG-RSP, SBC-REQ, SBC-RSP, PKM-REQ, PKM-RSP, REG-REQ, REG-RSP messages)
\item Control plane: Service flow creation/mgmt (DSA-REQ, DSA-RSP, DSA-ACK)
\item Control plane: Scanning (SCN-REQ, SCN-RSP)
\item Control plane: Handover (BSHO-REQ, MSHO-RSP, HO-IND)
\item Several traffic models: fixed, handover after timeout, distance based handover
\item Simple event signalling, so you don't have to understand whole
  stack operation to get some notification (similar to MIH
  802.21-style events, but easier)
\item Several optimization allowed by 802.16-2005 standard are
  configurable (you can enable or disable them)
\item multiple SSes support (optimizations can be enabled on a per SS basis)
\item Connection management and queuing
\item multiple BSes support
\item Handover between BSes can be simulated easily
\end{itemize}

On top of WiMAX layer, there is a working IPv6 stack. As primary
purpose of this simulation environment is focused automatic
configuration and DHCPv6 in particular, related areas are the most
developed:

\begin{itemize}
\item IPv6Node: traffic source/sink, with statistics and various traffic generation models
\item RAGen/RArcv: Router Advertisement generator and receiver, used to configure IPv6 nodes in stateless mode
\item DHCPv6 client: implementation of the DHCPv6 client, with several proposed enhancements
\item DHCPv6 server: as you probably guessed, it is used for providing configuration. Supports relays and several extra enhancements
\item MobileIPv6 Mobile Node: Simple implementation of the MN
\item MobileIPv6 Home Agent
\item IPv6 dispatcher, that inteligently redirects packets to specific
  modules. It is also mobility aware.
\end{itemize}

To better model complex environment, this simulation also provides
event based FSM (Finite State Machine) implementation. For each state,
there are up to 3 functions: onEnter(), onEvent() and onExit(). States
can be transitive or stationary. Also there's a well defined list of
inputs for each FSM.

Numbat code is available under GNU GPL (version 2 or later)
licence. It means that it can be downloaded, compiled, used, 
modified and even redistributed by all users, including commercial
purposes. Numbat uses the Omnet++ as simulation environment
\cite{omnet}\footnote{Omnet++ is distributed under other license. See
  its homepage for details.}
