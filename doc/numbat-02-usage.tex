\section{Installation and usage}

Numbat can be used in Windows and Linux systems. Ways of installation
are different. As Numbat is not a end user software, but rather
research environment, you need to gain a bit of expertise to use
it. Do not expect to have download and ready to use product anytime
soon.

\subsection{Linux compilation}

\subsubsection{Requirements}
To compile and run Numbat you will need properly installed Omnet++
(tested with version 3.3). Omnet++ packages and installation notes are
available on the Omnet Community website:
\href{http://omnetpp.org}{http://omnetpp.org}.

\subsubsection{Installation}
First you need to obtain Numbat sources from
{\href{http://klub.com.pl/projects/numbat/}{http://klub.com.pl/projects/numbat/}. You
  can download latest snapshot or sources directly from svn
  repository.

Before compilation you must prepare Makefile, execute command
opp\_makemake may be used. However, as sources are distributed over
serval directories, it is better to use provided script. Then make all
targets by simply running make:
\begin{lstlisting}
./rebuild-makefiles
make
\end{lstlisting}

We suppose that there weren't any errors, so you can run and enjoy simulation:
\begin{lstlisting}
cd wimax
./wimax
\end{lstlisting}

\subsection{Windows compilation}
Note: This section is based on document ``Omnet++ installation and
configuration guide in Windows® environment.'' by Andrzej Bojarski.

This guide shows how to install and configure Omnet++ in Windows
environment. This guide is mainly based on official Omnet++ manual,
but also contains some knowledge based on author's own experiences. It
does not cover all possible installation scenarios. Though it should
be treated as helpful source rather than complete guide.

First thing to consider is that Windows version of Omnet++ is
optimized for and should be used with Visual C++. As full version it
is a commercial product. However there are free versions like Visual
Studio 2005 Express Edition which are fully sufficient for Omnet++
uses. Also installing Platform SDK, which is available free too, is
necessary.

Omnet++ comes as an ordinary .exe installation file so it is obvious
for every windows user how to make use of it. Only thing that needs
explanation is environment selection. You have to choose VC++ release
you are using. If it's Visual C++ 2003 (7.1) you choose vc-71m, if
it's Visual C++ 2005 (8.1) you choose vc-81. It is not recommended to
use Visual Studio 6.0 because some of Omnet++ extensions have problem
to compile with it.

Installation directory should not have blank spaces as it may cause
compilation problems.

\subsubsection{Pre-compilation setup}
Before making any operations using Omnet++ it is obligatory to set
right system variables. They must be set for every new command line
window used for compilation. In other words, first thing You have to
do after opening the command line window is setting the system
variables. It is done by executing vcvarsall.bat or vcvars32.bat files
from VC++ installation directories. These files can usually found in
\texttt{C:$\backslash$Program Files$\backslash$Microsoft Visual Studio
8$\backslash$VC} or similar (depending on VC++ release) directory. It
is a good practice to make a .bat file executing system variables
setting to save time wasted for walking through directories.

On this stage Omnet++ should be fully functional, however sometimes
there is a compilation and/or linker problem with libraries
placement. That was my case. Moving all VC++ libraries to Omnet++
include and lib directories turned out to be helpful. However there
are other problems with configuration based on library compatibility
issues. If you get such errors, visit
http://www.omnetpp.org/pmwiki/index.php?n=Main.OmnetppInstallation to
get latest istallation tips. Main advice for such incompatibility is
to change version of VC++ as some versions tend to work incorrectly
with Omnet++.

\subsubsection{Actual compilation}
To create Makefile.vc which is VC++ file used for further compilation
run following command:

\begin{lstlisting}
\your\working\directory>opp\nmakemake -f
\end{lstlisting}
 
This will create new Makefile.vc, overwriting previous existing. Next,
compile files into executable:

\begin{lstlisting}
\your\working\directory>nmake -f Makefile.vc
\end{lstlisting}

Assuming you won't get any errors, .exe file should be made.
