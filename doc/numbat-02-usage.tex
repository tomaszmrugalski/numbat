\section{Installation and usage}

Numbat can be used in Windows and Linux systems. Ways of installation are different, but both quite simple.

\subsection{Linux installation}

\subsubsection{Requirements}
To compile and run Numbat you will need properly installed Omnet++ (tested with version 3.3). Omnet++ packages and installation notes are available on the Omnet Community website: \href{http://omnetpp.org}{http://omnetpp.org}.

\subsubsection{Installation}
First you need to obtain Numbat sources from {\href{http://klub.com.pl/projects/numbat/}{http://klub.com.pl/projects/numbat/}. You can download latest snapshot or sources directly from svn repository.

Before compilation you must prepare Makefile, execute command:
\begin{verbatim}
opp_makemake -f
\end{verbatim}

To build Numbat, type:
\begin{verbatim}
make
\end{verbatim}

We suppose that there weren't any errors, so you can run and enjoy simulation:
\begin{verbatim}
./wimax
\end{verbatim}
