\section{Installation and usage}

Numbat can be used in Windows and Linux systems. Ways of installation
are different. As Numbat is not a end user software, but rather
research environment, you need to gain a bit of expertise to use
it. Do not expect to have download and ready to use product anytime
soon.

\subsection{Linux compilation}

\subsubsection{Requirements}
To compile and run Numbat you will need properly installed Omnet++
(tested with version 3.3). Omnet++ packages and installation notes are
available on the Omnet Community website:
\href{http://omnetpp.org}{http://omnetpp.org}.

\subsubsection{Installation}
First you need to obtain Numbat sources from
{\href{http://klub.com.pl/projects/numbat/}{http://klub.com.pl/projects/numbat/}. You
  can download latest snapshot or sources directly from svn
  repository.

Before compilation you must prepare Makefile, execute command
opp\_makemake may be used. However, as sources are distributed over
serval directories, it is better to use provided script. Then make all
targets by simply running make:
\begin{lstlisting}
./rebuild-makefiles
make
\end{lstlisting}

We suppose that there weren't any errors, so you can run and enjoy simulation:
\begin{lstlisting}
cd wimax
./wimax
\end{lstlisting}

\subsection{Windows compilation}
Note: Windows compilation has been greatly simplified since migration
to Omnet-4.0. Currently it is quite straightforward process.

\begin{enumerate}
\item Run omnet shell (double click on mingwenc.cmd from Omnetpp-4.0 directory)
\item Change directory to numbat (cd path/to/numbat)
\item Rebuild makefiles (use command: rebuild-makefiles)
\item Build Numbat (make)
\item Run Numbat (omnet)
\end{enumerate}

